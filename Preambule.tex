%!Mode:: "TeX:UTF-8"
%!TEX program = lualatex
\usepackage[dvipsnames, table]{xcolor}     % to have colors 
\usepackage{eso-pic}                % put things into background 
\usepackage{lipsum}                 % for sample text
\usepackage[explicit]{titlesec}     % for customizing sections
\usepackage[leqno]{amsmath}         % for mathematical content
\usepackage[amsmath]{ntheorem}      % for theorem-like environments
\usepackage{esint}                  % for circulation integrals
\usepackage[framemethod=TikZ]{mdframed}

%Mes packages

\usepackage{imakeidx}
\DeclareGraphicsExtensions{.pdf,.png,.jpg}
%\usepackage{index}
% \makeindex[name=persons,title=Index des \textsc{Noms}]
\usepackage[totoc]{idxlayout}
\usepackage[utf8]{luainputenc}
%\usepackage[pdftex]{graphicx}
\usepackage[french]{babel}%Pour les écritures en français
\usepackage[T1]{fontenc}
\usepackage{amsmath}
% \usepackage[thmmarks]{ntheorem} % Pour les \qed
\usepackage{amsfonts}
\usepackage{amssymb}
\usepackage{mathrsfs} % Certains caractères de maths en plus, comme le T d'une tribu en proba
%\usepackage{amsthm}

%\usepackage{fancyhdr}
%\pagestyle{fancy}
%\fancyhf{} % FancyHDR has to be loaded before the geometry package
\usepackage[bottom=2.5cm, top=3cm, left=4.35cm, right=1.65cm]{geometry}
\usepackage{titlesec}
%\usepackage[usenames,dvipsnames]{color}%Package des couleurs avant celui de tikz
\usepackage{calc} %Pour les calculs d'entêtes de définitions
\usepackage{upgreek} % Pour le beau \varphi
\usepackage{tikz}
\usepackage{tikz-3dplot} %Pour les dessins en 3D
% \usepackage{pst-solides3d} % Pour des meilleurs dessins en 3D
\usetikzlibrary{arrows,shapes,trees,calc,through,patterns,positioning,intersections} % loads some tikz extensions

\catcode`\@=11 
\tikzset{rgb color/.code={\pgfutil@definecolor{.}{RGB}{#1}\tikzset{color=.}}} 
\tikzset{rgb fill/.code={\pgfutil@definecolor{.}{RGB}{#1}\tikzset{fill=.}}} 

\usepackage[european resistor, european voltage, european current]{circuitikz} % De http://www.physagreg.fr/schemas-figures-physique-svg-tikz.php
\usetikzlibrary{decorations.markings,decorations.pathmorphing,
decorations.pathreplacing}
\usetikzlibrary{shapes.geometric}

\usepackage{mathtools}    % Pour pouvoir rendre les "underbrace" moins encombrants en \mathclap
\usepackage{subfig}       % Pour les sous figures
\usepackage{needspace}    % Pour réserver de l'espace
\renewcommand{\thefootnote}{\fnsymbol{footnote}} %
                          % Pour que footcite utilise plutot des etoiles
\usepackage{caption}      % Pratique pour avoir des légendes sans "figure"
\usepackage{pifont}       % Pour cocher des cases
\usepackage{multirow}     % Tableaux avec cellules fusionnées
\usepackage{tabu}         % Pour varier les lignes dans les tableaux - http://tex.stackexchange.com/questions/41758/how-can-i-reproduce-this-table-with-thick-lines
\usepackage[babel=true]{csquotes}
\usepackage{enumitem}     % Pour personnaliser les listes
\frenchbsetup{StandardLists=true}
                          % Pour éviter tout conflit avec [enumitem]
\usepackage{multicol}     % Environnements multi-collonnes dans le document

%\usepackage[sfdefault]{ClearSans}    % Pour le texte
                          % Pour utiliser les guillemets Français
%\usepackage[europeancurrents,europeanresistors]{circuitikz}
                          % Pour dessiner des circuits électriques
\usepackage{wrapfig}      % Pour "wrap" des figures avec du texte
%\usepackage{enumerate}    % Pour numeroter en (i), (ii), (iii) etc...
\usepackage{esvect}       % Pour des vecteurs plus beaux
\usepackage{ulem}         % Pour pouvoir couper des phrases soulignées avec \ul
\usepackage[version=3]{mhchem}
                          % Pour les formules de chimie
\usepackage{siunitx}      % Pour les unités SI 
\sisetup{inter-unit-product = \ensuremath { { } \cdot { } } }
\usepackage{ifdraft}      % Pour des affichages dépendant du mode brouillon (Draft)
%\usepackage{acronym}     % Pour les acronymes
\usepackage{cancel}       % Pour barrer des parties de formules
\usepackage{colortbl}     % Pour définir le style des colonnes dans les tableaux
\usepackage{diagbox}      % Pour des cellules de tableau divisées par la diagonale
\usepackage{tabulary}     % Pour des tableaux de taille exacte, mais beaux
%\usepackage{showframe}   % Pour afficher les marges
\usepackage{pdflscape}    % Pour avoir quelques pages en mode paysage (utiliser \begin{landscape})
\usepackage{blkarray}     % Pour les matrices par bloc
\ifdraft{
    \usepackage{float}
    \floatstyle{boxed}
    \restylefloat{figure}
}{}
\usepackage[section]{placeins}     % Pour les barrières de Float (\FloatBarrier)
\usepackage[obeyFinal]{todonotes}
                          % Pour les TODO en marges
\reversemarginpar         % Pour avoir les TODO du bon coté
\usepackage[acronym,toc,shortcuts]{glossaries}
\usepackage{glossary-longragged}
\usepackage[unicode=true,bookmarks=true,colorlinks=false]{hyperref}
\makeindex
%\makeglossaries
\makenoidxglossaries
    \newacronym{TRC}{TRC}{Théorème de la Résultante Cinétique}
    \newacronym{TMC}{TMC}{Théorème du Moment Cinétique}
    \newacronym{PFD}{PFD}{Principe Fondamental de la Dynamique}
    \newacronym[longplural={Ondes Planes Progressives}]{OPP}{OPP}{Onde Plane Progressive}
    \newacronym[longplural={Ondes Planes Progressives Monochromatiques}]{OPPM}{OPPM}{Onde Plane Progressive Monochromatique}
    \newacronym[longplural={Séries à Termes Positifs}]{SATP}{SATP}{Série À Termes Positifs}
    \newacronym[longplural={Amplificateurs Opérationnels}]{AO}{AO}{Amplificateur Opérationnel}
    \newacronym{ARQS}{ARQS}{Approximation des Régimes Quasi-Stationnaires}
    \newacronym[longplural={Ondes Électromagnétiques}]{OEM}{OEM}{Onde Électromagnétique}
    \newacronym{FEM}{FEM}{Force Électromotrice}
%    \newacronym{cad}{CÀD}{C'est-à-dire}
    \newacronym[plural={DLs}, longplural={Développements Limités}]{DL}{DL}{Développement Limité}
    \newacronym[longplural={Espaces Vectoriels}]{ev}{ev}{Espace Vectoriel}
    \newacronym[longplural={Espaces Vectoriels Normés}]{evn}{evn}{Espace Vectoriel Normé}
    \newacronym[longplural={Lois de Compositions Internes}]{LCI}{LCI}{Loi de Composition Interne}
    \newacronym[longplural={Développements en Série Entière}]{DSE}{DSE}{Développement en Série Entière}
    \newacronym[longplural={Sous-Espaces Propres}]{sep}{s-ep}{sous-espace propre}


            \makeatletter

            \newif\ifpgf@circuit@germanvoltage
            \ctikzset{voltage/german/.code = {\pgf@circuit@germanvoltagetrue } }

            %% Output routine for generic bipoles

            \def\pgf@circ@drawvoltagegeneric{
                \pgfextra{
                    \ifnum \ctikzvalof{mirror value}=-1
                                    \ifpgf@circuit@bipole@voltage@below\pgf@circuit@bipole@voltage@belowfalse\else\pgf@circuit@bipole@voltage@belowtrue\fi
                    \fi

                    \ifpgf@circuit@bipole@voltage@below
                        \def\pgf@circ@voltage@angle{90}
                    \else
                        \def\pgf@circ@voltage@angle{-90} 
                    \fi 

                    \edef\pgf@temp{/tikz/circuitikz/bipoles/\pgfkeysvalueof{/tikz/circuitikz/bipole/kind}/voltage/distance from node}
                    \pgfkeysifdefined{\pgf@temp}
                        { \edef\distacefromnode{\ctikzvalof{bipoles/\pgfkeysvalueof{/tikz/circuitikz/bipole/kind}/voltage/distance from node}} }
                        { \edef\distacefromnode{\ctikzvalof{voltage/distance from node}} }
                    \edef\pgf@temp{/tikz/circuitikz/bipoles/\pgfkeysvalueof{/tikz/circuitikz/bipole/kind}/voltage/bump b}
                    \pgfkeysifdefined{\pgf@temp}
                        { \edef\bumpb{\ctikzvalof{bipoles/\pgfkeysvalueof{/tikz/circuitikz/bipole/kind}/voltage/bump b}} }
                        { \edef\bumpb{\ctikzvalof{voltage/bump b}} }
                }

                coordinate (pgfcirc@mid) at ($(\tikztostart) ! \distacefromnode ! (\ctikzvalof{bipole/name}.left)$)
                coordinate (pgfcirc@Vfrom) at ($(pgfcirc@mid) ! -\ctikzvalof{voltage/distance from line}\pgf@circ@Rlen ! \pgf@circ@voltage@angle:(\ctikzvalof{bipole/name}.left)$) 

                coordinate (pgfcirc@mid) at ($(\tikztotarget) ! \distacefromnode ! (\ctikzvalof{bipole/name}.right)$)
                coordinate (pgfcirc@Vto) at ($(pgfcirc@mid) ! \ctikzvalof{voltage/distance from line}\pgf@circ@Rlen ! \pgf@circ@voltage@angle : (\ctikzvalof{bipole/name}.right)$)

                \ifpgf@circuit@bipole@voltage@below
                    coordinate (pgfcirc@Vcont1) at ($(\ctikzvalof{bipole/name}.center) ! \bumpb ! (\ctikzvalof{bipole/name}.-110)$)
                    coordinate (pgfcirc@Vcont2) at ($(\ctikzvalof{bipole/name}.center) ! \bumpb ! (\ctikzvalof{bipole/name}.-70)$)
                \else
                    coordinate (pgfcirc@Vcont1) at ($(\ctikzvalof{bipole/name}.center) ! \bumpb ! (\ctikzvalof{bipole/name}.110)$)
                    coordinate (pgfcirc@Vcont2) at ($(\ctikzvalof{bipole/name}.center) ! \bumpb ! (\ctikzvalof{bipole/name}.70)$)
                \fi

                \ifpgf@circuit@germanvoltage
                  \ifpgf@circuit@bipole@voltage@below
                    coordinate (pgfcirc@Vcont1) at ($(\ctikzvalof{bipole/name}.center) ! \ctikzvalof{voltage/bump a} ! (\ctikzvalof{bipole/name}.-120)$)
                    coordinate (pgfcirc@Vcont2) at ($(\ctikzvalof{bipole/name}.center) ! \ctikzvalof{voltage/bump a} ! (\ctikzvalof{bipole/name}.-60)$)
                \else
                    coordinate (pgfcirc@Vcont1) at ($ (\ctikzvalof{bipole/name}.center) ! \ctikzvalof{voltage/bump a} ! (\ctikzvalof{bipole/name}.120)$)
                    coordinate (pgfcirc@Vcont2) at ($ (\ctikzvalof{bipole/name}.center) ! \ctikzvalof{voltage/bump a} ! (\ctikzvalof{bipole/name}.60)$)
                  \fi
                \fi

                \ifpgf@circuit@europeanvoltage
                    \ifpgf@circuit@germanvoltage
                      \ifpgf@circuit@bipole@voltage@backward
                        (pgfcirc@Vcont2)  -- node[currarrow, sloped,  allow upside down, pos=1] {} (pgfcirc@Vcont1)
                      \else
                        (pgfcirc@Vcont1)  -- node[currarrow, sloped,  allow upside down, pos=1] {} (pgfcirc@Vcont2)
                      \fi
                    \else
                      \ifpgf@circuit@bipole@voltage@backward
                        (pgfcirc@Vto) .. controls (pgfcirc@Vcont2)  and (pgfcirc@Vcont1) .. 
                            node[currarrow, sloped,  allow upside down, pos=1] {} 
                        (pgfcirc@Vfrom) 
                      \else
                        (pgfcirc@Vfrom) .. controls (pgfcirc@Vcont1)  and (pgfcirc@Vcont2) ..
                            node[currarrow, sloped,  allow upside down, pos=1] {}
                        (pgfcirc@Vto)   
                      \fi  
                    \fi      
                \else
                    \ifpgf@circuit@bipole@voltage@backward
                        (pgfcirc@Vfrom) node[inner sep=0, anchor=\pgf@circ@bipole@voltage@label@anchor]{\scriptsize$+$}   
                        (pgfcirc@Vto) node[inner sep=0, anchor=\pgf@circ@bipole@voltage@label@anchor]{$-$}
                    \else
                        (pgfcirc@Vfrom) node[inner sep=0, anchor=\pgf@circ@bipole@voltage@label@anchor]{\scriptsize$-$}   
                        (pgfcirc@Vto) node[inner sep=0, anchor=\pgf@circ@bipole@voltage@label@anchor]{$+$}
                    \fi 
                \fi
            }
            \makeatother

\author{\bsc{Alexandre Jouandin}}
\title{Mathématiques en MP*}

\hypersetup{pdftitle={Cours de Mathématiques},
 pdfauthor={Alexandre Jouandin},
 pdfsubject={MP*}}

\newcommand{\HRule}{\rule{\linewidth}{0.5mm}}
\newcommand{\E}{\ensuremath{\vv{E}}}
\newcommand{\B}{\ensuremath{\vv{B}}}
\renewcommand{\emph}[1]{\textbf{\textcolor{Cerulean}{#1}}}
\renewcommand{\d}{\mathrm{d}}

\newcommand{\titre}[1]{\hfill \\[1.5\baselineskip] \begin{Large} %%% Les titres dans les méthodes
\textbf{\textcolor{couleurFonce}{#1}}
\end{Large}\\[\baselineskip]}

%%%% Définition des théorèmes et définitions
%\newmdtheoremenv[<mdframed−options>]{<envname>}[<numberedlike>]{<caption>}[<within>]

% customize the tag form of the equations
\makeatletter
   \let\mytagform@=\tagform@
   \def\tagform@#1{\maketag@@@{\hbox{\llap{(\ignorespaces\textbf{\textcolor{RoyalBlue}{#1}}\unskip\@@italiccorr)\hspace{0.5\oddsidemargin}}}}\kern1sp} 
   \renewcommand{\eqref}[1]{{\mytagform@{\textcolor{BlueViolet}{\ref{#1}}}}}

\newcommand{\Attention}{\hfill \\
\hbox{\llap{(\ignorespaces \textbf{\textsc{Attention}}\unskip\@@italiccorr)\hspace{0.5\oddsidemargin}}}} %Le label Attention ! 

\makeatother

\usepackage{empheq}				   % for math boxes

% Definition des couleurs

% Couleurs du document
\definecolor{couleurClaire}{RGB}{210,230,255}  %Claire
\definecolor{couleurNoirClair}{RGB}{66,66,66}     %Noir clair
\definecolor{couleurGrisClair}{RGB}{233,233,233}  %Gris clair
\definecolor{couleurGrisFonce}{RGB}{188,188,188}  %Gris foncé
\definecolor{couleurFonce}{RGB}{25,71,86}    %Foncé

% 5 couleurs de dessins
\definecolor{couleur1}{RGB}{0,160,176} % Invisible UFO - Bleu
\definecolor{couleur3}{RGB}{106,74,60} % Carribic Brown - Marron
\definecolor{couleur4}{RGB}{204,51,63} % Caribic Red - Rouge
\definecolor{couleur2}{RGB}{235,104,65} % Carribic Sun - Orange
\definecolor{couleur5}{RGB}{255,204,0} % Bee - Jaune

\definecolor{couleurImp}{rgb}{1,0,0}
%%%%% Palette actuelle : %%%%%%%%%%%%%
% http://www.colourlovers.com/palette/148712/Gamebookers [Modifié]
%%%%% Autres : %%%%%%%%%%%%%%%%%%%%%%%
% http://www.colourlovers.com/palette/1673019/Docu-Fantasia
% http://www.colourlovers.com/palette/38562/Hands_On

% Les fonctions associées : 
\newcommand{\ibox}[1]{\fcolorbox{couleurNoirClair}{couleurClaire!20}{#1}}
\newcommand{\emphl}[1]{\textcolor{couleurGrisFonce}{#1}} % Light
\newcommand{\emphi}[1]{\textcolor{couleurGrisClair!90!Black}{#1}} % Index
\newcommand{\emphh}[1]{\textbf{\textcolor{couleurNoirClair}{#1}}} % Bold
\newcommand{\emphhs}[1]{\emphh{\uline{#1}}} % Bold Souligné

%\renewcommand{\CancelColor}{couleurFonce} % Du package cancel

\newcolumntype{x}{>{\columncolor{Gray}}c}
\newcolumntype{y}{>{\columncolor{white}}c}
\newcolumntype{b}{>{\bfseries{}}r}

  \numberwithin{equation}{chapter}
% \numberwithin{equation}{section}

%%%%%%%%%%%%%%%%%%%%%%%%%%%%%%%%%%%%%%%%%% DEFINITION %%%%%%%%
\newcounter{dfn}[part] 
\newenvironment{dfn}[1][]{%
\refstepcounter{dfn}% 
\ifstrempty{#1}% 
{\mdfsetup{%
frametitle={%
	\tikz[baseline=(current bounding box.east),outer sep=0pt]
	\node[anchor=east,rectangle,fill=white]
{\strut Définition~\thedfn};}} }%
{\mdfsetup{% 
	frametitle={%
\tikz[baseline=(current bounding box.east),outer sep=0pt] \node[anchor=east,rectangle,fill=white]
{\strut Définition~\thedfn : ~#1};}}%
}% 
\mdfsetup{skipbelow=10pt,leftmargin=-10mm, innerleftmargin=1mm, linecolor=black!80, outerlinewidth=1pt, topline=true, rightline=false, bottomline=false, frametitleaboveskip=-10pt} 
\begin{mdframed}[]\relax%
}{\end{mdframed}} 



%%%%%%%%%%%%%%%%%%%%%%%%%%%%%%%%%%%%%%%%%%%% THEOREME %%%%%%%%
\newcounter{theo}[part] 
\numberwithin{theo}{dfn}
\newenvironment{theorem}[1][]{%
\refstepcounter{theo}% 
\ifstrempty{#1}% 
{\mdfsetup{%
frametitle={%
	\tikz[baseline=(current bounding box.east),outer sep=0pt]
	\node[anchor=east,rectangle,fill=white]
{\strut Théorème~\thetheo};}} }%
{\mdfsetup{% 
	frametitle={%
\tikz[baseline=(current bounding box.east),outer sep=0pt] \node[anchor=east,rectangle,fill=white]
{\strut Théorème~\thetheo:~#1};}}%
}% 

\mdfsetup{ frametitleaboveskip=-10pt, leftmargin=0cm, innerleftmargin=2mm,leftline=true, linecolor=black!40,%
linewidth=1pt,topline=true,rightline=true}%
\begin{mdframed}[]\relax%
}{\end{mdframed}}


%%%%%%%%%%%%%%%%%%%%%%%%%%%%%%%%%%%%% THEOREME IMPORTANT %%%%
\newenvironment{itheorem}[1][]{%
\refstepcounter{theo}% 
\ifstrempty{#1}% 
{\mdfsetup{%
frametitle={%
	\tikz[baseline=(current bounding box.east),outer sep=0pt]
	\node[anchor=east,rectangle,color=couleurGrisFonce,fill=couleurFonce]
{\strut \textcolor{couleurClaire}{Théorème~\thetheo}};}} }%
{\mdfsetup{% 
	frametitle={%
\tikz[baseline=(current bounding box.east),outer sep=0pt] \node[anchor=east,rectangle,color=couleurGrisFonce,fill=couleurFonce]
{\strut \textcolor{couleurClaire}{Théorème~\thetheo:}\textcolor{White}{~#1}};}}%
}% 
\mdfsetup{ frametitleaboveskip=-10pt, leftmargin=0cm, innerleftmargin=2mm,leftline=true,outerlinewidth=0.5pt, linecolor=couleurGrisFonce, linewidth=1pt, topline=true, backgroundcolor=couleurGrisClair!40, rightline=true}%
\begin{mdframed}[]\color{couleurNoirClair}\relax%
%
}{\end{mdframed}}

\newmdtheoremenv[
  topline=false,
  rightline=false,
  bottomline=true,
  leftmargin=1cm,
  skipabove=0pt,
  skipbelow=1.5\medskipamount,
  footnoteinside=false
]{proof}{Preuve}[theo]

%%%%%%%%%%%%%%%%%%%%%%%%%%%%%%%%%%%%%%%%%%%% Lemme %%%%%%%%
\newenvironment{lemme}[1][]{% 
\ifstrempty{#1}% 
{\mdfsetup{%
frametitle={%
	\tikz[baseline=(current bounding box.east),outer sep=0pt]
	\node[anchor=east,rectangle,fill=white]
{\strut Lemme};}} }%
{\mdfsetup{% 
	frametitle={%
\tikz[baseline=(current bounding box.east),outer sep=0pt] \node[anchor=east,rectangle,fill=white]
{\strut Lemme~#1};}}%
}% 
\mdfsetup{skipbelow=10pt,skipabove=10pt,leftmargin=0.5cm,
innertopmargin=10pt,linecolor=Black!40,%
linewidth=1pt,topline=true,rightline=true,
frametitleaboveskip=-10pt} 
\begin{mdframed}[]\relax%
}{\end{mdframed}} 


%%%%%%%%%%%%%%%%%%%%%%%%%%%%%%%%%%%%%%%%%%%% METHODE %%%%%%%%
\newcounter{methode}[part]
\newenvironment{methode}[1][]{%
\refstepcounter{methode}% 
\ifstrempty{#1}% 
{\mdfsetup{%
frametitle={%
	\tikz[baseline=(current bounding box.east),outer sep=0pt]
	\node[anchor=east,rectangle,color=couleurFonce,fill=couleurNoirClair]
{\strut \textcolor{White}{Méthode}};}} }%
{\mdfsetup{% 
	frametitle={%
\tikz[baseline=(current bounding box.east),outer sep=0pt] \node[draw, anchor=east,rectangle,thick, color=couleurFonce,fill=couleurNoirClair]
{\strut Méthode:~#1};}}%
}% 
\mdfsetup{skipabove=10pt, leftmargin=-0.5cm, backgroundcolor=couleurClaire!15, linecolor=couleurFonce,
linewidth=2pt, topline=true, rightline=true, frametitleaboveskip=-10pt} 
\begin{mdframed}[]\relax%
}{\end{mdframed}} 



%%%%%%%%%%%%%%%%%%%%%%%%%%%%%%%%%%%%%%%%%%%% Exemple %%%%%%%%
\newcounter{exemple}[dfn]
\newenvironment{exemple}[1][]{%
\refstepcounter{dfn}% 
\ifstrempty{#1}% 
{\mdfsetup{%
frametitle={%
  \tikz[baseline=(current bounding box.east),outer sep=0pt]
  \node[anchor=east,rectangle,color=couleurFonce,fill=couleurGrisClair]
{\strut \textcolor{White}{Exemple}};}} }%
{\mdfsetup{% 
  frametitle={%
\tikz[baseline=(current bounding box.east),outer sep=0pt] \node[draw, anchor=east,rectangle,thick, color=couleurFonce,fill=couleurGrisClair]
{\strut Exemple:~#1};}}%
}% 
\mdfsetup{skipabove=10pt, leftmargin=0.5cm, linecolor=couleurFonce,
linewidth=1pt, topline=true, rightline=true, frametitleaboveskip=-10pt} 
\begin{mdframed}[]\relax%
}{\end{mdframed}} 


%%%%%%%%%%%%%%%%%%%%%%%%%%%%%%%%%%%%%%%%%%%% Propriétés %%%%%%%%
\newenvironment{prop}[1][]{%
\ifstrempty{#1}% 
{\mdfsetup{%
frametitle={%
	\tikz[baseline=(current bounding box.east),outer sep=0pt]
	\node[anchor=east,rectangle]
{\strut Propriétés};}} }%
{\mdfsetup{% 
	frametitle={%
\tikz[baseline=(current bounding box.east),outer sep=0pt] \node[anchor=east,rectangle]
{\strut Propriétés~#1};}}%
}% 
\mdfsetup{skipbelow=10pt,skipabove=10pt,leftmargin=0cm,
innerleftmargin=0.5mm,
innertopmargin=10pt,
linecolor=black!40,%
outerlinewidth=0.2mm,
topline=false,rightline=false, bottomline=false,
frametitleaboveskip=0} 
\begin{mdframed}[]\relax%
}{\end{mdframed}} 

\everymath{\displaystyle}

%Mes macros

\usepackage{xifthen} % Utile pour les newcommand et autres
% cf. : http://tex.stackexchange.com/a/58629

\newenvironment{changemargin}[2]{% Change Margin pour les équations de Maxwell
\begin{list}{}{%
\setlength{\topsep}{0pt}%
\setlength{\leftmargin}{#1}%
\setlength{\rightmargin}{#2}%
\setlength{\listparindent}{\parindent}%
\setlength{\itemindent}{\parindent}%
\setlength{\parsep}{\parskip}%
}%
\item[]}{\end{list}}

%%%% Commandes pour écriture facile —————————————————————————————————————
\newcommand{\Epsilon}{\ensuremath{\mathcal{E}}}
\newcommand{\ssi}{\emphhs{si et seulement si} }

%%%% Commandes pour Maths ———————————————————————————————————————————————
\newcommand{\Reel}{\ensuremath{\mathbb{R}}}
\newcommand{\Cmplx}{\ensuremath{\mathbb{C}}}
\newcommand{\CM}{\ensuremath{\mathcal{CM}}}
\newcommand{\reference}[1]{\textit{cf.} équation \eqref{#1} page \pageref{#1}}
\newcommand{\tr}{\mathrm{tr}}
\newcommand{\qed}{\hfill \ensuremath{\Box}} % Petit workaround de \qed...
\renewcommand{\epsilon}{\varepsilon} % Je me trompe tout le temps... 
\newcommand{\Grad}[1][]{\mathrm{Grad}\ifstrempty{#1}{\,}{\left( #1 \right)}} % Le même qu'en physique, mais sans vecteur : pas besoin en maths
%%%% Commandes pour mise en forme type HTML —————————————————————————————


%%%% Commandes pour Physique ————————————————————————————————————————————
    % Opérateurs 
\newcommand{\diverg}[1][]{\mathrm{div}\ifstrempty{#1}{\,}{\left( #1 \right)} }
\newcommand{\rot}[1][]{\vv{\mathrm{rot}}\ifstrempty{#1}{\,}{\left( #1 \right)} }
\newcommand{\grad}[1][]{\vv{\mathrm{grad}}\ifstrempty{#1}{\,}{\left( #1 \right)} }
\newcommand{\laplacien}[1][]{\Delta\ifstrempty{#1}{\,}{\left( #1 \right)} }

    % Optique Ondulatoire
\newcommand{\vibration}[1][]{\ensuremath{s_{#1}(M,t)}}
\newcommand{\amplitude}[2][]{\ensuremath{\textcolor{Periwinkle}{A_{#1}\left( #2 \right) }}}
\newcommand{\pulsation}[1][]{\ensuremath{\omega_{#1}{}}}
\newcommand{\phase}[2][]{\ensuremath{\textcolor{ProcessBlue}{\upvarphi_{#1}\left( #2 \right) }}}
\newcommand{\Phase}[2][]{\ensuremath{\textcolor{ProcessBlue}{\Phi_{#1}\left( #2 \right) }}}

\newcommand{\vibrationC}[1][]{\ensuremath{\underline{s_{#1}(M,t)}}}
\newcommand{\amplitudeC}[2][]{\ensuremath{\textcolor{Periwinkle}{\underline{a_{#1}\left( #2 \right) }}}}

% Autres macros de PhysiqueANCIEN
\newcommand{\der}[2]{\ensuremath{\dfrac{\d #1}{\d #2}}}
%\newcommand{\torseur}[4]{\ensuremath{\overrightarrow{\mathcal{#1}_\textbf{#3}}=\overrightarrow{#3_\textbf{#2}}+\overrightarrow{\textbf{#2#3}} \wedge \textbf{#4}}}%Torseur défini comme :
%Torseur{fonction}{point1}{point2}{Résultante}

%autres trouvées sur Internet
%\tikzset{every picture/.style=remember picture}%Sert à utiliser les noms globaux
%\newcommand{\mathnode}[2]{\mathord{\tikz[baseline=(#1.base), inner sep = 0pt]{\node (#1) {$#2$};}}}
% Define "struts" as suggested by Claudio Beccari in
% a piece in TeX and TUG News, Vol. 2, 1993.
\newcommand\Tstrut{\rule{0pt}{2.6ex}}       % "top" strut
\newcommand\Bstrut{\rule[-0.9ex]{0pt}{0pt}} % "bottom" strut
\newcommand{\TBstrut}{\Tstrut\Bstrut} % top&bottom struts

% \makeatletter % To use with empheq -> display eq ref
% \MHInternalSyntaxOn
% \renewenvironment{empheq}[2][]{%
%   \let\savedmaketag\maketag@@@
%   \renewcommand\eqref[1]{\textup{%
%       \let\maketag@@@\savedmaketag%
%       \tagform@{\ref{##1}}}%
%   }
%   \setkeys{EmphEqEnv}{#2}\setkeys{\EQ_options_name:}{#1}%
%   \EmphEqMainEnv}{\endEmphEqMainEnv}
% \MHInternalSyntaxOff
% \makeatother

%%%<
\usepackage{verbatim}
%%%>

\begin{comment}
:Title: Periodic Table of Chemical Elements

\end{comment}

\usepackage{ifpdf}
\usepackage{tikz}
\usepackage[active,tightpage]{preview}
\usetikzlibrary{shapes,calc}

\ifpdf
  %
\else
  % Implement Outline text using pstricks if regular LaTeX->dvi->ps->pdf route
  \usepackage{pst-all}
\fi
\usepackage{pdflscape}
